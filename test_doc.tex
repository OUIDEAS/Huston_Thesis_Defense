\documentclass[numbered,pdftex]{ohio-etd}

\usepackage[square,sort&compress,numbers]{natbib}
\usepackage[margin=1in]{geometry}
\usepackage{textcomp} 
\usepackage{amssymb}  
\usepackage{bm}       
\usepackage{booktabs}
\usepackage{dcolumn}  
\usepackage{multirow} 
\usepackage{ragged2e}
\usepackage{graphicx} 
\usepackage{float}
\usepackage{rotating}
\usepackage{makecell}
\usepackage{multirow}
\usepackage[english]{babel}
\usepackage{graphicx}
\usepackage{subcaption}
\usepackage{listings}
\usepackage[printonlyused]{acronym}

\justifying


% -- Basic formatting
\usepackage[utf8]{inputenc}
\usepackage[english]{babel}
\usepackage{times}
\setlength{\parindent}{8pt}
\usepackage{indentfirst}% -- Defining colors:
\usepackage[dvipsnames]{xcolor}
\definecolor{codegreen}{rgb}{0,0.6,0}
\definecolor{codegray}{rgb}{0.5,0.5,0.5}
\definecolor{codepurple}{rgb}{0.58,0,0.82}
\definecolor{backcolour}{rgb}{0.95,0.95,0.92}
\definecolor{light-gray}{gray}{0.95}

\lstset{basicstyle=\linespread{1.1}\ttfamily\footnotesize,
	backgroundcolor=\color{light-gray},
	xleftmargin=0.7cm,
	frame=tlbr, 
	framesep=0.2cm, 
	framerule=0pt,
	commentstyle=\color{codepurple},
	keywordstyle=\color{NavyBlue},
	numberstyle=\tiny\color{codegray},
	stringstyle=\color{codepurple},
	breakatwhitespace=false,         
	breaklines=true,                 
	captionpos=t,                    
	keepspaces=false,                 
	numbers=left,                    
	numbersep=2pt,                  
	showspaces=false,                
	showstringspaces=false,
	showtabs=false,                  
	tabsize=2,
	xleftmargin=\parindent
}


\begin{document}

{Velodyne rotational speed is required to determine the dt for determining the appropriate data set per 360\degree sweep. Each 360\degree sweep is read consecutively in a loop. }

\begin{lstlisting}[language=Matlab]
for struct_idx = 1:length(velodyne_packets_struct)

dT_loop                 = dT * struct;

% Extracting Point Clouds
%       timeDuration_points      = veloReader_points.StartTime + seconds(dT);
timeDuration_packets    = veloReader_packets.StartTime + seconds(dT_loop);

% Read first point cloud recorded
%       ptCloudObj_points        = readFrame(veloReader_points, timeDuration_packets);
ptCloudObj_packets      = readFrame(veloReader_packets, timeDuration_packets);

% Access Location Data
%     ptCloudLoc_points        = ptCloudObj_points.Location;
ptCloudLoc_packets      = ptCloudObj_packets.Location;

% Access Intensity Data
ptCloudInt_packets      = ptCloudObj_packets.Intensity;

...\end{lstlisting}				

{With the appropriate data selected, each of the Velodyne VLP-32C's 32 channels are individually examined for Cartesian point coordinates and intensity values.}

\begin{lstlisting}[language=Matlab]
% Extracting data
for j = 1:32

x                       = ptCloudLoc_packets(j,:,1);
y                       = ptCloudLoc_packets(j,:,2);
z                       = ptCloudLoc_packets(j,:,3);
int                     = ptCloudInt_packets(j,:);

x_append                = [x_append x];
y_append                = [y_append y];
z_append                = [z_append z];
int_append              = [int_append int];

end % Extracting data

XYZI_TOT                = [x_append' y_append' z_append'];

end\end{lstlisting}	

	
	%\chapter{Work Plan}
	%{
		%	
		%	{Here be a worthless gantt chart!}
		%
		%}
	
	
	%
	%%% The \references command marks the point where the bibliography will
	%%% be inserted.  In the example below, the BibTeX commands are
	%%% inserted into the macro definition, but one could also manually
	%%% enter the bibliography entries if desired.
	%\references{
		%  %ETD does not define a particular required bibliography style.
		%  %They DO require entries to end in a period UNLESS they end in a URL
		%  %or DOI AND that DOIs be set in the same font as everything else.
		%  %I have modified the alphaurl style file, to conform to these requirements.
		%  
		\bibliographystyle{ieeetr}  
		\bibliography{resources.bib} 
		%}
	
	%%\appendix           % Indicates the transition from chapters to appendices.  
	% Subsequent 
	
	%%\chapter{An Appendix}
	%%\section{A Section in the Appendix}
	
\end{document}
